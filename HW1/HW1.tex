\documentclass[]{article}
\usepackage{lmodern}
\usepackage{graphicx}
\usepackage{adjustbox}
\usepackage{amssymb,amsmath}
\usepackage{ifxetex,ifluatex}
\usepackage{listings}
  \usepackage[T1]{fontenc}
  \usepackage[utf8]{inputenc}
\usepackage{microtype}
\usepackage[margin=1in]{geometry}
\usepackage{hyperref}
\usepackage{framed}
\usepackage{graphicx,grffile}
\makeatletter
\def\maxwidth{\ifdim\Gin@nat@width>\linewidth\linewidth\else\Gin@nat@width\fi}
\def\maxheight{\ifdim\Gin@nat@height>\textheight\textheight\else\Gin@nat@height\fi}
\makeatother
% Scale images if necessary, so that they will not overflow the page
% margins by default, and it is still possible to overwrite the defaults
% using explicit options in \includegraphics[width, height, ...]{}
\setkeys{Gin}{width=\maxwidth,height=\maxheight,keepaspectratio}
\setlength{\parindent}{0pt}
\setlength{\parskip}{6pt plus 2pt minus 1pt}
\setlength{\emergencystretch}{3em}  % prevent overfull lines
\providecommand{\tightlist}{%
  \setlength{\itemsep}{0pt}\setlength{\parskip}{0pt}}

%%% Change title format to be more compact
\usepackage{titling}

% Create subtitle command for use in maketitle
\newcommand{\subtitle}[1]{
  \posttitle{
    \begin{center}\large#1\end{center}
    }
}

\setlength{\droptitle}{-2em}
  \title{MSAN 691 - Homework 1}
  \pretitle{\vspace{\droptitle}\centering\huge}
  \posttitle{\par}
  \author{Christine Chu, Andre Guimaraes Duarte, Mikaela Hoffman-Stapleton, April Wang}
  \preauthor{\centering\large\emph}
  \postauthor{\par}
  \predate{\centering\large\emph}
  \postdate{\par}
  \date{September 1, 2016}
  
% Redefines (sub)paragraphs to behave more like section*s
\ifx\paragraph\undefined\else
\let\oldparagraph\paragraph
\renewcommand{\paragraph}[1]{\oldparagraph{#1}\mbox{}}
\fi
\ifx\subparagraph\undefined\else
\let\oldsubparagraph\subparagraph
\renewcommand{\subparagraph}[1]{\oldsubparagraph{#1}\mbox{}}
\fi

\usepackage{color}

%%%%%%%%%%%%%%%%%%%%%%%%%%%%%%%%%%%%%%%%%%%%%%%%%%%%%%%%%%%%%%%%%%%%%%%%%%%%%%%%%%%%%%%%%%%%%%%%%%%%%%%%%%%%%%%%%%%%%%%
\begin{document}
\maketitle

%%%%%%%%%%%%%%%%%%%%%%%%%%%%%%%%%%%%%%%%%%%%%%%% question 3 %%%%%%%%%%%%%%%%%%%%%%%%%%%%%%%%%%%%%%%%%%%%%%%%%%%%%%%%%%%
\section*{Question 3}
%%%%%%%%%%%%%%%%%%%%%%% a %%%%%%%%%%%%%%%%%%%%%%%%%%
\paragraph{a)}
In order to get all columns relating to CUSIP 45920010, we use the following query:

\color{blue}
\begin{verbatim}
SELECT *
FROM stocks2016.d2010
WHERE cusip = '45920010';
\end{verbatim}
\color{black}

We get 252 resulting rows, of which we only show 10 in this write-up:

\begin{adjustbox}{width=1.2\linewidth,totalheight=4in, center}
\begin{tabular}{|l | r | r | r | r | l | r | r | r | r | l | r|}
\hline
\textit{cusip} & \textit{permno} & \textit{permco} & \textit{issuno} & \textit{hsic} & \textit{retdate} & \textit{bid} & \textit{ask} & \textit{prc} & \textit{vol} & \textit{ret} & \textit{shrout} \\
\hline
45920010 & 12490 & 20990 & 0 & 7379 & 2010-01-04 & 130.85001 & 132.97 & 132.45 & 6155800 & 0.011841 & 1313603 \\
45920010 & 12490 & 20990 & 0 & 7379 & 2010-01-05 & 130.10001 & 131.85001 & 130.85001 & 6842500 & -0.012080 & 1313603 \\
45920010 & 12490 & 20990 & 0 & 7379 & 2010-01-06 & 129.81 & 131.49001 & 130 & 5605300 & -0.006496 & 1313603 \\
45920010 & 12490 & 20990 & 0 & 7379 & 2010-01-07 & 128.91 & 130.25 & 129.55 & 5840600 & -0.003462 & 1313603 \\
45920010 & 12490 & 20990 & 0 & 7379 & 2010-01-08 & 129.05 & 130.91991 & 130.85001 & 4197100 & 0.010035 & 1313603 \\
45920010 & 12490 & 20990 & 0 & 7379 & 2010-01-11 & 128.67 & 131.06 & 129.48 & 5731200 & -0.010470 & 1313603 \\
45920010 & 12490 & 20990 & 0 & 7379 & 2010-01-12 & 129 & 131.33 & 130.50999 & 8083400 & 0.007955 & 1313603 \\
45920010 & 12490 & 20990 & 0 & 7379 & 2010-01-13 & 129.16 & 131.12 & 130.23 & 6458300 & -0.002145 & 1313603 \\
45920010 & 12490 & 20990 & 0 & 7379 & 2010-01-14 & 129.91 & 132.71001 & 132.31 & 7114500 & 0.015972 & 1313603 \\
45920010 & 12490 & 20990 & 0 & 7379 & 2010-01-15 & 131.089 & 132.89 & 131.78 & 8502300 & -0.004006 & 1313603 \\
\ldots & \ldots & \ldots & \ldots & \ldots & \ldots & \ldots & \ldots & \ldots & \ldots & \ldots & \ldots \\
\hline
\end{tabular}
\end{adjustbox}

\begin{center}
\noindent (252 rows) \\
\end{center}

%%%%%%%%%%%%%%%%%%%%%%% b %%%%%%%%%%%%%%%%%%%%%%%%%%
\paragraph{b)}
In order to get all columns relating to CUSIP 45920010 on the 7th of January, we use the following query:

\color{blue}
\begin{verbatim}
SELECT *
FROM stocks2016.d2010
WHERE cusip = '45920010'
  AND retdate = '2010-01-07';
\end{verbatim}
\color{black}

We get 1 resulting row:

\begin{adjustbox}{width=1.2\linewidth,totalheight=4in, center}
\begin{tabular}{|l | r | r | r | r | l | r | r | r | r | l | r|}
\hline
\textit{cusip} & \textit{permno} & \textit{permco} & \textit{issuno} & \textit{hsic} & \textit{retdate} & \textit{bid} & \textit{ask} & \textit{prc} & \textit{vol} & \textit{ret} & \textit{shrout} \\
\hline
45920010 & 12490 & 20990 & 0 & 7379 & 2010-01-07 & 128.91 & 130.25 & 129.55 & 5840600 & -0.003462 & 1313603 \\
\hline
\end{tabular}
\end{adjustbox}

\begin{center}
\noindent (1 row) \\
\end{center}

%%%%%%%%%%%%%%%%%%%%%%% c %%%%%%%%%%%%%%%%%%%%%%%%%%
\paragraph{c)}
In order to get the difference between the bid and the ask for CUSIP 45920010 on the 7th of January, we use the query:

\color{blue}
\begin{verbatim}
SELECT bid-ask AS difference
FROM stocks2016.d2010
WHERE cusip = '45920010'
  AND retdate = '2010-01-07';
\end{verbatim}
\color{black}

We get one resulting value for the difference between bid and ask:

\begin{center}
\begin{tabular}{|r|}
\hline
\textit{difference} \\
\hline
-1.34 \\
\hline
\end{tabular}

\noindent (1 row) \\
\end{center}

%%%%%%%%%%%%%%%%%%%%%%% d %%%%%%%%%%%%%%%%%%%%%%%%%%
\paragraph{d)}
In order to get the days when CUSIP 45920010 has a volume less than 5 million and a bid over \$140, we use the query:

\color{blue}
\begin{verbatim}
SELECT retdate
FROM stocks2016.d2010
WHERE cusip = '45920010'
  AND vol < 5000000
  AND bid > 140;
\end{verbatim}
\color{black}

We get 31 resulting dates, of which we only the show 10 in this write-up:

\begin{center}
\begin{tabular}{|l|}
\hline
\textit{retdate} \\
\hline
2010-11-02 \\
2010-11-05 \\
2010-11-08 \\
2010-11-11 \\
2010-11-12 \\
2010-11-15 \\
2010-11-17 \\
2010-11-18 \\
2010-11-22 \\
2010-11-23 \\
\ldots \\
\hline
\end{tabular}

\noindent (31 rows) \\
\end{center}


%%%%%%%%%%%%%%%%%%%%%%%%%%%%%%%%%%%%%%%%%%%%%%%% question 4 %%%%%%%%%%%%%%%%%%%%%%%%%%%%%%%%%%%%%%%%%%%%%%%%%%%%%%%%%%%
\section*{Question 4}
%%%%%%%%%%%%%%%%%%%%%%% a %%%%%%%%%%%%%%%%%%%%%%%%%%
\paragraph{a)}
In order to get the market capitalization for permno 14593 on the first of February in 2010, we use the following query:

\color{blue}
\begin{verbatim}
SELECT shrout*prc AS market_capitalization
FROM stocks2016.d2010
WHERE permno = 14593
  AND retdate = '2010-02-01';
\end{verbatim}
\color{black}

We get one resultin market capitalization value:

\begin{center}
\begin{tabular}{|r|}
\hline
\textit{market\_capitalization} \\
\hline
176580190.35 \\
\hline
\end{tabular}

\noindent (1 row) \\
\end{center}

%%%%%%%%%%%%%%%%%%%%%%% b %%%%%%%%%%%%%%%%%%%%%%%%%%
\paragraph{b)}
In order to get the permnos and market capitalizations for the companies with the top 5 market capitalization-days in 2010, we use the following queries:

\color{blue}
\begin{verbatim}
SELECT permno, shrout*prc AS market_capitalization
FROM stocks2016.d2010
WHERE shrout*prc IS NOT null
ORDER BY market_capitalization DESC
LIMIT 5;
\end{verbatim}
\color{black}

We get 5 resulting rows containing the permno and the market capitalization:

\begin{center}
\begin{tabular}{|r | r|}
\hline
\textit{permno} & \textit{market\_capitalization} \\
\hline
11850 & 370224534.94 \\
11850 & 369972407.09 \\
11850 & 369921981.52 \\
11850 & 369115172.4 \\
11850 & 368711767.84 \\
\hline
\end{tabular}

\noindent (5 rows) \\
\end{center}

%%%%%%%%%%%%%%%%%%%%%%% c %%%%%%%%%%%%%%%%%%%%%%%%%%
\paragraph{c)}
In order to get the permnos and market capitalizations for the companies with the top 5 market capitalizations on February 3rd 2010, we use the following query:

\color{blue}
\begin{verbatim}
SELECT permno, shrout*prc AS market_capitalization
FROM stocks2016.d2010
WHERE shrout*prc IS NOT null
  AND retdate = '2010-02-03'
ORDER BY market_capitalization DESC
LIMIT 5;
\end{verbatim}
\color{black}

We get 5 resulting rows containing the permno and the market capitalization:

\begin{center}
\begin{tabular}{|r | r|}
\hline
\textit{permno} & \textit{market\_capitalization} \\
\hline
11850 & 315144406.8 \\
10107 & 251098298.43 \\
55976 & 206778034.44 \\
18163 & 181972751.45 \\
14593 & 180660767.85 \\
\hline
\end{tabular}

\noindent (5 rows) \\
\end{center}

%%%%%%%%%%%%%%%%%%%%%%% d %%%%%%%%%%%%%%%%%%%%%%%%%%
\paragraph{d)}
In order to get the permnos and market capitalizations for the companies with the bottom 5 market capitalization-days in 2010, we use the following queries:

\color{blue}
\begin{verbatim}
SELECT permno, shrout*prc AS market_capitalization
FROM stocks2016.d2010
ORDER BY market_capitalization ASC
LIMIT 5;
\end{verbatim}
\color{black}

We get 5 resulting rows containing the permno and the market capitalization:

\begin{center}
\begin{tabular}{|r | r|}
\hline
\textit{permno} & \textit{market\_capitalization} \\
\hline
88335 & -5641098.965 \\
79977 & -4748698.4 \\
91462 & -1674922.53 \\
88811 & -1356934.75 \\
83264 & -1276178.53 \\
\hline
\end{tabular}

\noindent (5 rows) \\
\end{center}

%%%%%%%%%%%%%%%%%%%%%%% e %%%%%%%%%%%%%%%%%%%%%%%%%%
\paragraph{e)}
In order to get the permnos and market capitalizations for the companies with the bottom 5 market capitalizations on February 3rd 2010 with stocks that have a trading volume of less than 10 million, we use the following query:

\color{blue}
\begin{verbatim}
SELECT permno, shrout*prc AS market_capitalization
FROM stocks2016.d2010
WHERE retdate = '2010-02-03'
  AND vol < 10000000
ORDER BY market_capitalization ASC
LIMIT 5;
\end{verbatim}
\color{black}

We get 5 resulting rows containing the permno and the market capitalization:

\begin{center}
\begin{tabular}{|r | r|}
\hline
\textit{permno} & \textit{market\_capitalization} \\
\hline
61508 & -484627.92 \\
92394 & -484163.35 \\
29014 & -402096.45 \\
91278 & -314288.595 \\
90228 & -242844.48 \\
\hline
\end{tabular}

\noindent (5 rows) \\
\end{center}


%%%%%%%%%%%%%%%%%%%%%%%%%%%%%%%%%%%%%%%%%%%%%%%% question 5 %%%%%%%%%%%%%%%%%%%%%%%%%%%%%%%%%%%%%%%%%%%%%%%%%%%%%%%%%%%
\section*{Question 5}
%%%%%%%%%%%%%%%%%%%%%%% a %%%%%%%%%%%%%%%%%%%%%%%%%%
\paragraph{a)}
In order to get the permno of the company with the smallest bid-ask spread in 2010 with a volume less than $25,000$, we use the following query:

\color{blue}
\begin{verbatim}
SELECT permno
FROM stocks2016.d2010
WHERE vol < 25000
ORDER BY abs(bid-ask) ASC
LIMIT 1;
\end{verbatim}
\color{black}

We get one resulting permno:

\begin{center}
\begin{tabular}{|r|}
\hline
\textit{permno} \\
\hline
10001 \\
\hline
\end{tabular}

\noindent (1 row) \\
\end{center}

%%%%%%%%%%%%%%%%%%%%%%% b %%%%%%%%%%%%%%%%%%%%%%%%%%
\paragraph{b)}
In order to get the permno of the company with the smallest bid-ask spread on February 8th 2010 that also had more than $500,000$ shares outstanding, we use the following query:

\color{blue}
\begin{verbatim}
SELECT permno
FROM stocks2016.d2010
WHERE retdate = '2010-02-08'
  AND shrout > 500000
ORDER BY abs(bid-ask) ASC
LIMIT 1;
\end{verbatim}
\color{black}

We get one resulting permno:

\begin{center}
\begin{tabular}{|r|}
\hline
\textit{permno} \\
\hline
75789 \\
\hline
\end{tabular}

\noindent (1 row) \\
\end{center}

%%%%%%%%%%%%%%%%%%%%%%% c %%%%%%%%%%%%%%%%%%%%%%%%%%
\paragraph{c)}
In order to find the permno and the bid-ask spread of the stock with the smallest bid-ask apread when the colume is less than $1,000$, we use the following query:

\color{blue}
\begin{verbatim}
SELECT permno, abs(bid-ask) AS spread
FROM stocks2016.d2010
WHERE vol < 1000
ORDER BY spread ASC
LIMIT 1;
\end{verbatim}
\color{black}

We get one resulting row with the permno and the bid-ask spread:

\begin{center}
\begin{tabular}{|r | r|}
\hline
\textit{permno} & \textit{spread} \\
\hline
10001 & 0 \\
\hline
\end{tabular}

\noindent (1 row) \\
\end{center}


%%%%%%%%%%%%%%%%%%%%%%%%%%%%%%%%%%%%%%%%%%%%%%%% question 6 %%%%%%%%%%%%%%%%%%%%%%%%%%%%%%%%%%%%%%%%%%%%%%%%%%%%%%%%%%%
\section*{Question 6}
%%%%%%%%%%%%%%%%%%%%%%% a %%%%%%%%%%%%%%%%%%%%%%%%%%
\paragraph{a)}
In order to get the company that had the highest net income, we use the following query:

\color{blue}
\begin{verbatim}
SELECT tic
FROM stocks2016.fnd
WHERE netinc IS NOT null
ORDER BY netinc DESC
LIMIT 1;
\end{verbatim}
\color{black}

We get one resulting tic:

\begin{center}
\begin{tabular}{|l|}
\hline
\textit{tic} \\
\hline
GM \\
\hline
\end{tabular}

\noindent (1 row) \\
\end{center}

%%%%%%%%%%%%%%%%%%%%%%% b %%%%%%%%%%%%%%%%%%%%%%%%%%
\paragraph{b)}
In order to get the company that had the highest net income in fiscal year 2011, we use the following query:

\color{blue}
\begin{verbatim}
SELECT tic
FROM stocks2016.fnd
WHERE fyear = 2011
  AND netinc IS NOT null
ORDER BY netinc DESC
LIMIT 1;
\end{verbatim}
\color{black}

We get one resulting tic:

\begin{center}
\begin{tabular}{|l|}
\hline
\textit{tic} \\
\hline
XOM \\
\hline
\end{tabular}

\noindent (1 row) \\
\end{center}

%%%%%%%%%%%%%%%%%%%%%%% c %%%%%%%%%%%%%%%%%%%%%%%%%%
\paragraph{c)}
In order to get the company which had the lowest net income, we use the following query:

\color{blue}
\begin{verbatim}
SELECT tic
FROM stocks2016.fnd
ORDER BY netinc ASC
LIMIT 1;
\end{verbatim}
\color{black}

We get one resulting tic:

\begin{center}
\begin{tabular}{|l|}
\hline
\textit{tic} \\
\hline
FNMA \\
\hline
\end{tabular}

\noindent (1 row) \\
\end{center}

%%%%%%%%%%%%%%%%%%%%%%% d %%%%%%%%%%%%%%%%%%%%%%%%%%
\paragraph{d)}
In order to get the company which had the lowest net income in fiscal year 2011, we use the following query:

\color{blue}
\begin{verbatim}
SELECT tic
FROM stocks2016.fnd
WHERE fyear = 2011
ORDER BY netinc ASC
LIMIT 1;
\end{verbatim}
\color{black}

We get one resulting tic:

\begin{center}
\begin{tabular}{|l|}
\hline
\textit{tic} \\
\hline
FNMA \\
\hline
\end{tabular}

\noindent (1 row) \\
\end{center}

%%%%%%%%%%%%%%%%%%%%%%% e %%%%%%%%%%%%%%%%%%%%%%%%%%
\paragraph{e)}
In order to get the company which had more than $1,000$ employees, had the highest net income per employee in 2011, we use the following query (note: emp is in thousands of employees):

\color{blue}
\begin{verbatim}
SELECT tic
FROM stocks2016.fnd
WHERE emp > 1
  AND fyear = 2011
  AND netinc IS NOT null
ORDER BY netinc/emp DESC
LIMIT 1;
\end{verbatim}
\color{black}

We get one resulting tic:

\begin{center}
\begin{tabular}{|l|}
\hline
\textit{tic} \\
\hline
CEO \\
\hline
\end{tabular}

\noindent (1 row) \\
\end{center}

%%%%%%%%%%%%%%%%%%%%%%% f %%%%%%%%%%%%%%%%%%%%%%%%%%
\paragraph{f)}
In order the find the company which had a net-income per employee over \$$1,000$ had the largest number of employees, we use the following query (note: emp is in thousands of employees, and netinc is in millions of dollars):

\color{blue}
\begin{verbatim}
SELECT tic
FROM stocks2016.fnd
WHERE emp > 0
  AND netinc/emp > 1
ORDER BY emp DESC
LIMIT 1;
\end{verbatim}
\color{black}

We get one resulting tic:

\begin{center}
\begin{tabular}{|l|}
\hline
\textit{tic} \\
\hline
WMT \\
\hline
\end{tabular}

\noindent (1 row) \\
\end{center}



\end{document}
